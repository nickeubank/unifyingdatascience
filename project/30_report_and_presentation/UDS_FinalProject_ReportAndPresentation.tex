\documentclass[12pt]{article}



\usepackage[T1]{fontenc}
\usepackage{amsfonts, amsmath, amssymb}
\usepackage{multirow}
\usepackage{epsfig}
\usepackage{subfigure}
\usepackage{subfloat}
\usepackage{graphicx}
\usepackage{hyperref}
\usepackage{parskip}
\usepackage{booktabs}
\usepackage{longtable}
\usepackage[utf8]{inputenc}
\usepackage[english]{babel}
% \usepackage[document]{ragged2e}
\usepackage{verbatim, rotating, paralist}
\usepackage{enumerate}

\usepackage{natbib}


\usepackage{pdfsync}
\usepackage{latexsym}
\usepackage{amsthm}
\usepackage{mathabx}

\usepackage{stmaryrd}
\usepackage{mathrsfs}
\usepackage{dsfont}
\usepackage{fancyhdr}
\usepackage{color}

\usepackage{parskip}
\usepackage{anysize, indentfirst, setspace}
\usepackage[right=2cm, left=2cm, top=3cm, bottom=3cm]{geometry}
\usepackage{appendix}

\usepackage{enumitem}
\setlist{nosep}

\title{Unifying Data Science Final Project \\ Report and Presentation Guidelines}
\author{Nick Eubank}



%-------------------------- BEGIN DOCUMENT ----------------------------------%
\begin{document}
\maketitle

Your final report should be written to an imaginary stakeholder whom you may assume to be very thoughtful but \emph{not} explicitly trained in data science. As a result, you will need to work hard to communicate your results, their strengths and limitations, and implications without reliance on technical language. 

In particular, your report should generally follow the structure suggested below (you may adjust the organization if you want, but you should definitely include everything listed): 

\begin{enumerate}
    \item \textbf{Identify your audience:}  At the top of your report, please specify the stakeholder to whom you are addressing your report -- a product manager, a legislative aid, a policymaker, etc. This stakeholder should be relevant to your study, but should not be someone with data science training. You may assume they know about basic statistical concepts (means and standard deviations), but no more (no assumed understanding of potential outcomes, the theoretical underpinnings of experiments, specific designs like differences-in-differences, etc.). Obviously this is not something you'd put in a real write-up, but will be helpful for evaluation of your project. 
    \item \textbf{Introduction / Executive Summary}
    \begin{itemize}
        \item \textbf{Identify the problem you wish to address:} The first thing to do in any report is \emph{motivate} your analysis -- tell us about \emph{why} you need to undertake this project. At this point in the report, keep this relatively brief -- the motivation for the project is important, but you don't want to drown the reader in background. This should probably be one-to-two solid paragraphs.  
        \item \textbf{What question will you try to answer, and how will it help you address :} Here's the linchpin of the report: announce the answer you're seeking to answer in your project \emph{and} make it clear how this will help address the problem you've identified. This transition is where you will either get the reader to buy into the report and read it carefully, or lose their interest. 
        \item \textbf{Summary your strategy:} Now in a couple paragraphs provide an overview of your project, your approach, and a preliminary summary of your results.
    \end{itemize}
    \item \textbf{Background}
    \begin{itemize}
        \item Now that you've hooked your reader, you can circle back and provide any additional background needed to help the reader better understand your motivation or the specific context you are analyzing (if you're looking at a policy change, the details of the policy, the context in which it occurred, the players involved, etc.)
    \end{itemize}
    \item \textbf{Your Design}
    \begin{itemize}
        \item Here's where you lay out how you plan to answer the question you laid out in your summary. Crucial to this section is to help the reader understand \emph{why} you're using a specific causal design \emph{without} just using technical language. To do so, you want to lay out a \emph{specific, concrete reason} that just using observational data might lead to erroneous conclusions (e.g. do the same thing you did on the homework assignments / midterm when asked about how people were interpreting observational studies.)\\
        For example, if you are doing an experiment to see how sending people coupons would impact consumer behavior, you want to explain that ``we can't just use data on sales from stores that chose to send out coupons to evaluate whether we should be sending out coupons to all our customers because it's possible that the stores that sent out coupons did so precisely because they knew that their customers were struggling financially, and thus needed coupons to be able to afford products. As a result, if we compared sales to customers who got coupons to those who did not, we might inadvertently assume the lower sales to customers who got coupons was the result of the coupons, when in fact it actually reflected the fact that the coupons went to customers who were less well-off financially to begin with.'' \\
        ``But if we run an experiment in which we randomly assign customers to either receive a coupon or not, then we know that on average the people getting coupons will be the same as the people not getting coupons (since who gets coupons is random, and not related to anything like customer income). As a result, we can compare sales to customers who got coupons and those that did not, and infer with confidence that any difference we see is the result of getting coupons, not other differences in the customers with or without coupons.'' \\
        (See? No discussion of potential outcomes or use of terms like ``baseline differences!'')
        \item Because you are writing to a non-technical audience, there may be technical considerations you don't need to include in this part of the report. In general, when you write to fellow data scientists, you're writing to convince them that you did everything correctly, and that your assumptions are air-tight. In other words, you're writing to a \emph{skeptical audience}. \\
        But a stakeholder reading your analysis will generally trust that you've done things correctly. After all, they often won't know the knowledge to evaluating things like the detailed statistics from a balance tests or A/A test. As a result, your job when writing to a stakeholder is to tell them \emph{what they need to know}, not to prove you did things right. That does include some things about what you've done -- they need to know why your causal research design is better than just looking at observational data in a regular regression, and any limitations of what you've learned -- but you don't have to put in the report every bit of due deligience you've done. 
    \end{itemize} 
\end{enumerate}



\end{document}
