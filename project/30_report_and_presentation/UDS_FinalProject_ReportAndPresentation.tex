\documentclass[12pt]{article}



\usepackage[T1]{fontenc}
\usepackage{amsfonts, amsmath, amssymb}
\usepackage{multirow}
\usepackage{epsfig}
\usepackage{subfigure}
\usepackage{subfloat}
\usepackage{graphicx}
\usepackage{hyperref}
\usepackage{parskip}
\usepackage{booktabs}
\usepackage{longtable}
\usepackage[utf8]{inputenc}
\usepackage[english]{babel}
% \usepackage[document]{ragged2e}
\usepackage{verbatim, rotating, paralist}
\usepackage{enumerate}

\usepackage{natbib}


\usepackage{pdfsync}
\usepackage{latexsym}
\usepackage{amsthm}
\usepackage{mathabx}

\usepackage{stmaryrd}
\usepackage{mathrsfs}
\usepackage{dsfont}
\usepackage{fancyhdr}
\usepackage{color}

\usepackage{parskip}
\usepackage{anysize, indentfirst, setspace}
\usepackage[right=2cm, left=2cm, top=3cm, bottom=3cm]{geometry}
\usepackage{appendix}

\usepackage{enumitem}
\setlist{nosep}

\title{Unifying Data Science Final Project \\ Report and Presentation Guidelines}
\author{Nick Eubank}



%-------------------------- BEGIN DOCUMENT ----------------------------------%
\begin{document}
\maketitle

Your final report should be written to an imaginary stakeholder whom you may assume to be very thoughtful but \emph{not} explicitly trained in data science. As a result, you will need to work hard to communicate your results, their strengths and limitations, and implications without reliance on technical language. 

In particular, your report should generally follow the structure suggested below (you may adjust the organization if you want, but you should definitely include everything listed): 

\begin{enumerate}
    \item \textbf{Identify your audience:}  At the top of your report, please specify the stakeholder to whom you are addressing your report -- a product manager, a legislative aid, a policymaker, etc. This stakeholder should be relevant to your study, but should not be someone with data science training. You may assume they know about basic statistical concepts (means and standard deviations), but no more (no assumed understanding of potential outcomes, the theoretical underpinnings of experiments, specific designs like differences-in-differences, etc.). Obviously this is not something you'd put in a real write-up, but will be helpful for evaluation of your project. 
    \item \textbf{Identify the problem you wish to address:} The first thing to do in any report is \emph{motivate} your analysis -- tell us about \emph{why} you need to undertake this project. At this point in the report, keep this relatively brief -- the motivation for the project is important, but you don't want to drown the reader in background. This should probably be one-to-two solid paragraphs.  
    \item \textbf{What question will you try to answer, and how will it help you address :} Here's the linchpin of the report: announce the answer you're seeking to answer in your project \emph{and} make it clear how this will help address the problem you've identified. This transition is where you will either get the reader to buy into the report and read it carefully, or lose their interest. 
    \item \textbf{Summary your strategy:} Now in a couple paragraphs provide an overview of your project, your approach, and a preliminary summary of your results. 
\end{enumerate}



\end{document}
