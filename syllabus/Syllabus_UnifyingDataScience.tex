\documentclass[12pt]{article}



\usepackage[T1]{fontenc}
\usepackage{amsfonts, amsmath, amssymb}
\usepackage{multirow}
\usepackage{epsfig}
\usepackage{subfigure}
\usepackage{subfloat}
\usepackage{graphicx}
\usepackage{hyperref}
\usepackage{parskip}
\usepackage{booktabs}
\usepackage{longtable}
\usepackage[utf8]{inputenc}
\usepackage[english]{babel}
% \usepackage[document]{ragged2e}
\usepackage{verbatim, rotating, paralist}
\usepackage{enumerate}

\usepackage{natbib}


\usepackage{pdfsync}
\usepackage{latexsym}
\usepackage{amsthm}
\usepackage{mathabx}

\usepackage{stmaryrd}
\usepackage{mathrsfs}
\usepackage{dsfont}
\usepackage{fancyhdr}
\usepackage{color}

\usepackage{parskip}
\usepackage{anysize, indentfirst, setspace}
\usepackage[right=2cm, left=2cm, top=3cm, bottom=3cm]{geometry}
\usepackage{appendix}

\usepackage{enumitem}
\setlist{nosep}

\renewcommand{\topfraction}{.85}
\renewcommand{\bottomfraction}{.7}
\renewcommand{\textfraction}{.15}
\renewcommand{\floatpagefraction}{.66}
\renewcommand{\dbltopfraction}{.66}
\renewcommand{\dblfloatpagefraction}{.66}




% \pagestyle{fancyplain}
% \rhead{\hfill \small \emph{MIDS NUMBER -- Fall 2019}}
\cfoot{}

% \renewcommand{\headrulewidth}{0pt}



%-------------------------- BEGIN DOCUMENT ----------------------------------%
\begin{document}


\singlespacing






%------------------------- HEADER ---------------------------------%
\thispagestyle{empty}
\begin{minipage}[t]{.5\textwidth}
	Nicholas Eubank \\
	 Assistant Research Professor\\
	 Zoom, \url{https://duke.zoom.us/my/nickeubank} \\
	 Office Hours: Monday 12:45-1:45
     \vspace*{0.1cm}
\end{minipage}
\begin{minipage}[t]{.5\textwidth}
	\begin{flushright}  IDS 701 \\
	Spring 2023\\
	Tuesday / Thursday 1:45-3:00 
    \vspace*{0.1cm}
\end{flushright}
\end{minipage}


% line
\line(1,0){499}

\vspace{.35in}

\begin{center}
	\textbf{\LARGE{Unifying Data Science: Using Data Science to Solve Problems} }
\end{center}







%--------------------------------------------COURSE DESCRIPTION--------------------------------------------------%

\section{Course Description}

All too often, students learn data science by taking a course on machine learning from a computer scientist, a course on statistical modelling from a statistician, and a course on causal inference from a social scientist. As a result, graduating students find themselves with a toolbox of techniques, but no clear idea of how to use them to solve problems.

The aim of this course is to overcome this fragmentation and to provide students with a unified approach to using data science to solve real world problems. To that end, we will introduce a question-first, backwards design framework for systematically designing a data science project. Through exercises, students will practice each step of this approach, from working with stakeholders to properly articulate the problem they are seeking to address, to picking a question (which, if answered, will help the stakeholder solve their problem), selecting the appropriate methodological approach to answering that question, and developing a concrete strategy for generating an answer.

Having established this framework for solving data science problems, the class will then pivot to providing an application-focused introduction to \emph{causal inference}, the art and science of using statistical data to make causal statements about the world. Our approach will be rooted in the potential outcomes framework, and will cover a range of methods of statistical inference including randomized experiments, pre-post analysis, differences-in-differences, and instrumental variables. In addition, we will also discuss concepts like the distinction between internal and external validity, and the limitations of estimating Average Treatment Effects.

Finally, towards the end of the semester—once we have covered causal inference in this class and MIDS students have covered machine learning in detail in IDS 705—we will return to our more general investigation of how best to use data science to solve problems, now with a focus on when (supervised) machine learning approaches are appropriate and when causal approaches are preferable.

In addition to completing a number of exercises related to project design, over the semester students will conduct a complete data science project themselves. Data science is a fundamentally applied field, and there is no substitute for learning to put these project design principles into action through practice. These projects will be developed incrementally over the course of the semester with instructor guidance. By the end of the semester, students will have picked a topic area, developed a (tractable) question, decided what an answer to that question would actually look like, developed a work plan for generating that answer, and executed and presented their project, and then iterated the project based on feedback from their initial presentation. For MIDS students, this will serve as a "capstone-project with training wheels" to prepare students for their second-year Capstone projects with external partners. And this project should provide all students with a portfolio piece they can present to potential future employers. 

Throughout the course, we will also be consistently returning to a few themes, chief among them the importance of developing a skeptical mindset. This is a core data science skill, but one that students do not always have the opportunity to practice. In this course, we will discuss *and practice* approaching our data, our code, our statistical models, our problem statements, and the work of others from a constructive but skeptical perspective.

\subsection{Pre-Requisites for Non-MIDS Students}

This course is primarily designed for students in the Duke Masters in Interdisciplinary Data Science (MIDS) program, but students from other programs are more than welcome if they have the appropriate pre-requisite training. Data Science is a fundamentally interdisciplinary field, so the more perspectives we have represented in the classroom the better!

This course will assume that enrolled students have a good grasp of inferential statistics and statistical modelling (e.g. a course in linear models), though no prior experience with causal inference is expected. In addition, MIDS students will be taking a concurrent course in applied machine learning, and so incoming students will also be expected to have some basic experience with machine learning, or be concurrently enrolled in an applied machine learning course.

This course will also assume students are comfortable manipulating real-world data in either Python or R. The substantive content of this course is language-independent, but because students will be required to work on their projects in teams, comfort with one of these two languages will be required to facilitate collaboration (MIDS students are, generally, "bilingual" in R and Python). Where code examples are provided in class, they will use Python (`pandas`), but both the instructor and our TAs are also capable of providing support in R.

Finally, students will also be expected to be comfortable collaborating using git and github. If you meet the other requirements for this course but are not familiar with git and github, this is a skill you should be able to pickup on your own in advance of the course without too much difficulty. You can read more about \href{https://www.practicaldatascience.org/html/git_and_github.html}{git and github here}. The \href{https://library.duke.edu/data/}{Duke Center for Data and Visualization Science} also hosts git and github workshops for Duke students.


%--------------------------------------------COURSE ASSIGNMENTS------------------------------------------------%
\section{Assignments \& Grading}

\subsection{Participation (20\% of Grade)}

A major component of good participation is good \emph{preparation}. Because we will often use class time for exercises, it is absolutely critical that students do their assigned readings before \emph{every} class. Students who do not work through the instructional materials they have been assigned before class will not only get very little out of in-class exercises designed to reinforce the assigned materials, but they will also undermine the learning of the students they are asked to work with. With that in mind, students who do not complete their assigned readings before every class should be expected to see this reflected in their participation grades.

\textbf{Cold calling:} In the interest of creating an interactive learning experience, I will often ``cold call'' students with questions about the material we are discussing. To be clear my goal with cold calling is not to ``catch'' students who haven't done the reading, but rather to ensure that everyone is getting an opportunity to participate in the discussion. However, students who regularly demonstrating \emph{unfamiliarity with readings} can expect to receive lower participation scores (not having the right answer will not get you a low score, to be clear! The material in this course is difficult, so I don't always expect everyone to have the right answers on the tip of their time, but it's pretty easy for an instructor to recognize the difference between somebody who is really wrestling with the material and a student who just hasn't done the reading). 

Participation will be graded as follows:

\textbf{A range.}  You are fully \emph{and consistently} engaged in class discussion and exercises.  You both listen and contribute actively.  You are well-prepared for class.  Having done more than merely read the material, you have spent time thinking \emph{carefully and deeply} about the material's relationship to other materials and ideas presented in previous classes. You are not only able to answer questions about the material, but also come to class with thoughtful questions.  When working in teams, you work \emph{with} your partner. If your partner is struggling with an exercise, you help them understand the material rather than just completing the material on your own. If you are struggling with material, you ask for help (both from the instructor and your fellow students) and do not simply lean on your partner to complete the exercise. \\

\textbf{B range.}  You are engaged in class discussion and exercises.  You listen and contribute regularly.  You come well-prepared to class having read the material and your contributions show your familiarity, but your level of engagement lacks the depth accumulated through extra time spent thinking about the material.  When working in teams, you work \emph{with} your partner when they have a similar level of understanding, but do not always invest in helping a struggling partner to understand the material. You often ask for help when you are struggling, but other times you let your partner just complete the exercise. \\

\textbf{C range.}  You have met the minimum requirements of participation.  You are usually, but not always prepared.  You participate sometimes, but not regularly.  The comments that you offer show a basic familiarity with the materials, but do not help to build a coherent or productive discussion.  When working in teams, you only sometimes work \emph{with} your partner. When your partner is struggling, you often just do the exercise yourself. If you are struggling, you often do not ask for help and allow your partner to take over the exercise. \\

\textbf{D range.}  You have not met the minimum requirements of participation.  You are unprepared for class.  You have not read with the material with sufficient engagement to know even the most basic elements.  When working in teams, you do not attempt to work \emph{with} your partner. When your partner is struggling, you just do the exercise yourself. If you are struggling, you do not ask for help and allow your partner to take over the exercise.\\

\textbf{As should be clear from this rubric, above all it is important to emphasize that participation is evaluated on the basis of \emph{quality} and \emph{consistently}, \emph{not} quantity. Moreover, when completing in-class exercises, good participation is not about finishing first or without ever asking for help; good participation in in-class exercises is about helping your partner understand the material, and asking for help when you need it.}

\subsection{Causal Inference Mid-Term (20\% of Grade)}

At the completion of the causal inference portion of our course we will have a mid-term exam. 

\subsection{Interim Assignments (20\% of Grade)}

Over the course of the semester, students will be asked to complete a number of small assignments as homework. These assignments will, in total, be worth 20\% of student grades.

\subsection{Reading Reflections (20\% of Grade)}

Both to provide the instructor and teaching assistants with information about what topics students have found difficult, and also to ensure that students are doing the required readings (a necessity for a flipped classroom designed to be effective), students will be required to submit answers to a set of prompts about the required readings by 9am on the morning of each class.

\subsection{Team Data Science Project (20\% of Grade)}

Over the course of the semester, you and your team will develop a full data science project—from conception to execution and presentation. Your scores on the various components of this project—including graded drafts, intermediate work, teamwork, and project management skills—will jointly constitute 20\% of your overall grade.

\subsection{Late Assignments, Make Up Exams and Extra Credit}

\textbf{Late Assignment}

All students get one ``freebie''—they may submit \emph{one} assignment \emph{one} day late without penalty.

Freebies may be used for team assignments, but only if all team members have a freebie to use, and all agree to use their freebie for the team assignment.

After that, because of the difficulty associated with managing late assignment in large classes, all late assignments will be penalized 10\% per day the assignment is late, up to a maximum penalty of 50\%. 

The final deadline for accepting assignments that are more than five school days late is at the discretion of the instructor, and may vary by assignment. 

Exceptions to these late penalties may be made for students dealing with exceptional circumstances (illness for themselves or family, etc.)—if you are dealing with a difficult situation, please feel free to contact me to discuss your situation.

% \section{Texts}
%
% We will rely on two primary texts for this course (both of which, thankfully, are reasonably priced):
%
% \begin{itemize}
% 	\item \href{https://www.amazon.com/Python-Data-Science-Handbook-Essential-dp-1491912057/dp/1491912057}{\emph{Python Data Science Handbook: Essential Tools for Working with Data}} by Jake VanderPlas. Referred to in the syllabus as JVP.
% 	\item \href{https://www.amazon.com/gp/product/1491957662}{\emph{Python for Data Analysis: Data Wrangling with Pandas, NumPy, and IPython, Second Edition}} by Wes McKinney. Referred to in the syllabus as WM. \\
% 	\textbf{Make sure to buy the Second Edition!}.
% \end{itemize}
%
% We will also do some readings from \href{https://www.amazon.com/Code-Language-Computer-Hardware-Software/dp/0735611319}{Code: The Hidden Language of Computer Hardware and Software} by Petzold, Charles. It's a fun book and not very expensive, but we won't use it a lot so copies of relevant chapters will be provided if you don't want to buy it.
%
% \section{Course Schedule}

% Because one aim of this course is to ensure that all MIDS students have a solid foundation for their time at Duke, the exact organization of this course is likely to change regularly as the course proceeds. Students will therefore be expected to regularly (i.e. before every class) check on the updated course schedule (which will include assignments for the next class) at \href{https://www.practicaldatascience.org}{www.practicaldatascience.org}.

% \section{Laptop Policy}

% The causal evidence from the teaching and learning field clearly shows that learning outcomes are worse when students have laptops in the classroom. This appears to be due in part to computers allowing students to become easily distracted (e.g. Facebook), and also in part because significant information synthesis occurs when we take notes by hand because we are not fast enough writers to transcribe everything being said.  Most typists, on the other hand, can conceivably transcribe word for word for what's being said, but this requires little to no mental processing of the information.\footnote{For excellent work on this topic as well as a nice summary of existing literature on the impact of laptops on learning outcomes, see \href{https://cpb-us-w2.wpmucdn.com/sites.udel.edu/dist/6/132/files/2010/11/Psychological-Science-2014-Mueller-0956797614524581-1u0h0yu.pdf}{Mueller and Oppenheimer (2014, \emph{The Pen Is Mightier Than the Keyboard: Advantages of Longhand Over Laptop  Note Taking})}. \href{https://www.winona.edu/psychology/media/friedlaptopfinal.pdf}{Fried (2007, \emph{In-class laptop use and its effects on student learning})} and  \href{https://www.ncbi.nlm.nih.gov/pubmed/28182528}{Ravizza et. al. (2017, \emph{Logged In and Zoned Out})} are also worth reading.}

% Moreover, research also makes clear that the presence of laptops in the classroom undermines learning not only for the person with the laptop, but also for other students in the room (presumably it is distracting to have the computer next to you jumping back and forth from instagram to facebook).

% For these reasons, during portions of the class when I am speaking and when you are not actively engaged in programming exercises, I do not allow laptops out in class. (When I use slides, those will be provided online so you can refer back to them later if you would like).

% However, please \textbf{do bring your laptops every day}, as we will use them for in-class quizzes, and some in-class exercises.

% \textbf{Note:} If you would like an exception to this rule for medical reasons or due to learning differences, please speak to me and/or follow the directions in the Disability Policy section below, and we will be sure to find an arrangement that works for you.

\section{Honor Policy}

Duke University is a community dedicated to scholarship, leadership, and service and to the principles of honesty, fairness, respect, and accountability. Citizens of this community commit to reflect upon and uphold these principles in all academic and nonacademic endeavors, and to protect and promote a culture of integrity.

Remember the \href{https://studentaffairs.duke.edu/conduct/about-us/duke-community-standard}{Duke Community Standard} that you have agreed to abide by:

\begin{itemize}
	\item I will not lie, cheat, or steal in my academic endeavors;
	\item I will conduct myself honorably in all my endeavors; and
	\item I will act if the Standard is compromised.
\end{itemize}

Cheating on exams or plagiarism on homework assignments, lying about an illness or absence and other forms of academic dishonesty are a breach of trust with classmates and faculty, violate the Duke Community Standard, and will not be tolerated. Such incidences will result in a 0 grade for all parties involved. Additionally, there may be penalties to your final class grade along with being reported to the MIDS program directors.

\section{Disability Policy}

In an effort to prevent students with disabilities from having to explain and justify their condition separately to each of their various instructors, Duke has centralized disability management in the \href{https://access.duke.edu/students}{Student Disabilities Access Office}. If you think there is a possibility you may need an accommodation during this course, please reach out to their office as soon as possible (processing can take a little time).

Medical information shared with the SDAO are strictly confidential, and if SDAO determines an accommodation is appropriate, faculty members will simply be informed of the accommodation they are required to provide, not the underlying medical reason for the accommodation.

If you have any problems with SDAO, please let me know as soon as possible.

\section{Final}

While this course does not have a final exam, we may use our ``final'' time slot for group presentations, so please keep it open.

\section{Mental Health and Wellness}

Mental health and wellness is of primary importance at Duke, and the university offers resources to support students in managing daily stress and self-care. Duke offers several resources for students to seek assistance on coursework and to nurture daily habits that support overall well-being, some of which are listed below:

\begin{itemize}
	\item The Academic Resource Center: (919) 684-5917,the ARC@duke.edu, or arc.duke.edu
	\item DuWell: (919) 681-8421, provides Moments of Mindfulness (stress management and resilience building) and Koru (meditation) programming to assist students in developing a daily emotional well-being practice. To see schedules for programs please see https://studentaffairs.duke.edu/duwell. All are welcome and no experience necessary. duwell@studentaffairs.duke.edu, or https://studentaffairs.duke.edu/duwell
\end{itemize}

If your mental health concerns and/or stressful events negatively affect your daily emotional state, academic performance, or ability to participate in your daily activities, many resources are available to help you through difficult times. Duke encourages all students to access these resources.

\begin{itemize}
	\item \textbf{DukeReach}.Provides comprehensive outreach services to identify and support students in managing all aspects of well-being. If you have concerns about a student's behavior or health visit the website for resources and assistance.http://studentaffairs.duke.edu/dukereach
	\item \textbf{Counseling and Psychological Services (CAPS)}. CAPS services include individual, group, and couples counseling services, health coaching, psychiatric services, and workshops and discussions.CAPS also provides referral to off-campus resources for specialized care.(919) 660-1000. https://studentaffairs.duke.edu/caps
	\item \textbf{Blue Devils Care}. A convenient, confidential, and free way for Duke students to receive 24/7 mental health support through TalkNow and scheduled counseling. bluedevilscare.duke.edu
	\item \textbf{Two-Click Support}. Duke Student Government and DukeReach partnership that connects students to help in just two clicks.  https://bit.ly/TwoClickSupport
\end{itemize}

\section{Attendance Policy Related to COVID Symptoms, Exposure, or Infection}

Student health, safety, and well-being are the university's top priorities. To help ensure your well-being and the well-being of those around you, please do not come to class if you have symptoms related to COVID-19, have had a known exposure to COVID-19, or have tested positive for COVID-19. If any of these situations apply to you, you must follow university guidance related to the ongoing COVID-19 pandemic and current health and safety protocols.  

If you are experiencing any COVID-19 symptoms, \href{https://coronavirus.duke.edu/if-you-feel-sick/}{contact student health.} 919-681-9355. To keep the university community as safe and healthy as possible, you will be expected to follow these guidelines. Please reach out to me and your academic dean as soon as possible if you need to quarantine or isolate so that we can discuss arrangements for your continued participation in class.  

\section{Student Signature}

I have read and understand this syllabus. \\


Name: \\
\vspace{2cm}\\
Signature:


\end{document}
