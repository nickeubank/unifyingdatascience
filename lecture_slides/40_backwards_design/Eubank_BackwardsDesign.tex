% So we make this "beamer" rather than document!

\documentclass[11pt]{beamer}
% For handout add ,handout after 11pt

\usetheme[sectionpage=none,numbering=none]{metropolis}           % Use metropolis theme
	% To do printouts, add ", handout"  after aspectratio.
\usepackage{booktabs}
\usepackage{graphicx}
\usepackage{color}

\title{Backwards Design \\ in Data Science \\ Causal Inference Edition}
\author{\small Nick Eubank}
\date{\vspace*{.3in} \date}


% This is the beginning of a real document!
\begin{document}


\begin{frame}
\maketitle
\end{frame}

\begin{frame}[c]{Backwards Design}

Approach to planning data science projects
  \begin{itemize}
    \pause \item (Though backwards design isn't unique to DS)
  \end{itemize}
Goals:
\begin{itemize}
  \item Minimize wasted effort
  \pause \item Make sure you develop explicit goals
  \begin{itemize}
    \item Not get lost in your tools and data
  \end{itemize}
\end{itemize}
\end{frame}

\begin{frame}[c]{Backwards Design}

  Start with where you want to end up, then work backwards

\end{frame}

\begin{frame}[c]{Backwards Design}
  \begin{enumerate}
    \pause \item Determine Problem / Topic Area
    \pause \item What \emph{causal question} are you seeking to answer?
    \pause \item What would your ideal experiment be?
    \pause \item Where can you (a) measure your outcome variable, and (b) find variation in your treatment variable?
    \pause \item What would a feasible design be?
    \pause \item What does an answer look like?
    \pause \item What variables do you need to generate that answer?
    \pause \item What data contains those variables?
  \end{enumerate}
\end{frame}

\begin{frame}[c]{Step 0: Define the \alert{Problem / Topic}}
Why are you doing this project? \\
\pause What \alert{motivates} your investigation?\\
\vspace*{0.1cm}
\pause Examples:
\begin{itemize}
  \item Polling places change locations all the time. Are there consequences for voters? Could this be manipulated for political gain?
  \item My company (Zarbucks) recently did major store renovations. Did they improve sales?
\end{itemize}
\end{frame}


\begin{frame}[c]{Step 1: What \alert{question} are you seeking to answer?}
The tools of data science are fundamentally designed to \alert{answer questions}, \pause so to before you pick your tools, you have to decide \alert{what question you wish to answer.}\\
\vspace*{0.1cm}
\pause $\Rightarrow$ The MOST important part of your project\\
\end{frame}

\begin{frame}[c]{Step 1: What \alert{question} are you seeking to answer?}
Most important because:

\begin{itemize}
  \pause \item if you can't define the question you are seeking to answer, \alert{you'll find yourself lost in your data}, or worse
  \pause \item after finishing your project, you'll realizing the question you answered doesn't help solve the problem that motivated you.
\end{itemize}
\pause  $\Rightarrow$ Invest in this stage of your project \emph{before} you dive into the data!
\end{frame}


\begin{frame}[c]{Step 1: What \alert{question} are you seeking to answer?}

A critical feature of a good question is that it is \emph{tractable} and \emph{answerable} in a data science project. \\
\begin{itemize}
  \item If your question does not directly imply a course of action in your data science project, it's too vague.
\end{itemize}
\end{frame}

\begin{frame}[c]{Step 1: What \alert{question} are you seeking to answer?}
Not answerable:
\begin{itemize}
  \item What happens when you move a polling place?
  \item Are Zarbucks renovations worth it?
\end{itemize}
\pause
Answerable:
\begin{itemize}
  \item What is the effect of moving a polling place on voter turnout?
  \item Do Zarbucks renovations result in increased sales?
\end{itemize}
\end{frame}

\begin{frame}[c]{Step 1: What \alert{question} are you seeking to answer?}
How do I know if my answer is answerable / tractable?

\begin{enumerate}
  \pause \item Can you hypothesize an answer to your question? \\
  i.e. Can you state what you think might be the answer to your question?
  \pause \item Can you imagine what the answer to your question looks like?
\end{enumerate}
\end{frame}



\begin{frame}[c]{Step 2: What would the \alert{ideal experiment} look like?}

Ignore feasibility -- if you were a god, what experiment would you want to run?
\pause
\vspace{1cm}
Why is this helpful?
\begin{itemize}
  \item Helps you think through what your \alert{treatment} is,
  \item What your outcome is, and
  \item What \emph{variation} in your treatment looks like.
\end{itemize}
Separately from worrying about \emph{feasibility}.
\end{frame}

\begin{frame}[c]{Step 2: What would the \alert{ideal experiment} look like?}
\begin{itemize}
  \item Randomly assign voters to have new polling places or not
  \item Randomly renovate stores and observe consequences
\end{itemize}
\end{frame}

\begin{frame}[c]{Step 3: Pick a study context}
Answering causal questions requires \alert{variation in treatment assignment}.
\pause
Need a place where you can:
\begin{itemize}
  \item Observe \alert{variation} in treatment assignment, and
  \item Measure your target outcome
\end{itemize}
\end{frame}

\begin{frame}[c]{Step 3: Pick a study context}
\begin{itemize}
  \item North Carolina: Polling places move, and North Carolina has public data on voter registration, polling place locations, and turnout
  \pause \item Zarbucks: If we worked for Zarbucks: was there variation in \alert{when} people had renovations? If not, can we get data from before and after renovations?
\end{itemize}
\end{frame}

\begin{frame}[c]{Step 4: What design might be feasible?}

In your \alert{ideal experiment}, you identified your ``treatment.''

Now you need to find \alert{variation} in your treatment somewhere in the real world.
\end{frame}

\begin{frame}[c]{Step 4: What design might be feasible?}
Once you find variation, decide what comparisons you want to make.

Common strategies:

\begin{itemize}
  \item Experiment (e.g. AB testing)
  \item Pre-Post (variation over time within a unit)
  \item Cross-sectional (variation across units)
  \item Differences-in-differences (variation across time and units)
\end{itemize}
\end{frame}


\begin{frame}[c]{Step 4: What design might be feasible?}
North Carolina
\begin{itemize}
  \item If data from one election, then cross-section (maybe matching?) \\
  Could use census data for community demographics
  \item If data from multiple elections, we can do a difference-in-difference
\end{itemize}
\pause
Zarbucks:
\begin{itemize}
  \item If all renovations happened at once, pre-post
  \item If variation in timing, difference-in-difference
  \pause \item If you have POWER, randomize rollouts
\end{itemize}
\end{frame}

\begin{frame}[c]{Step 5: What does \alert{an answer} to your question look like?}
\alert{Write down} what the answer to your question will look like!
\pause
\begin{itemize}
  \item A figure
  \item A table or regression
  \item A dataset with predicted values
\end{itemize}
\pause
$\Rightarrow$ Ask yourself: if I gave that to my stakeholder / put it in a paper, would people be pleased?\\
\pause (OK, they might want robustness, and extensions, but at its core, is this an answer?)
\end{frame}


\begin{frame}[c]{Step 4: What does \alert{an answer} to your question look like?}
But it's not enough to imagine \emph{one} answer. You should be able to imagine what an answer to your question looks like if your hypothesis \alert{is true} and the if your hypothesis \alert{is false}. \\
\pause Otherwise your question isn't falsifiable! \\
\vspace*{0.3cm}
Write down what your answer looks like if your hypothesis is true, \emph{and} if it's false!
\end{frame}


\begin{frame}[c]{Step 5: What do you need to generate that answer?}
  Congratulations! You've just specified the goal of your analysis! \\
  \pause In my view, that is actually the hardest part of being a good data scientist. \\
  \pause ...Though probably not the part that will take up the majority of your time.
\end{frame}

\begin{frame}[c]{Step 5: What do you need to generate that answer?}
  So you now have in mind a table you want to generate. What data and variables do you need to create that result?
  \pause
  \pause For each variable, specify:
  \begin{enumerate}
    \item What do you need the variable to measure?
    \item For what population do you need the variable defined?
  \end{enumerate}
\end{frame}

\begin{frame}[c]{Step 5: Where can you get those variables?}
  \begin{enumerate}
    \item Where can you get those variables?, and
    \item How will you relate your different datasets?
  \end{enumerate}
\end{frame}

\begin{frame}[c]{In Class}

You've been hired by an real estate agent industry group that wants to know if it should campaign against laws that require mandatory disclosure of problems with houses.

They aren't sure if mandatory disclosure will increase sales (since buyers will be less worried about hidden problems), or decrease sales (since more problems may be made evident to customers).

\end{frame}

\end{document}
