% So we make this "beamer" rather than document!

\documentclass[11pt,handout]{beamer}
% For handout add ,handout after 11pt

\usetheme[sectionpage=none,numbering=none]{metropolis}           % Use metropolis theme
	% To do printouts, add ", handout"  after aspectratio.
\usepackage{booktabs}
\usepackage{graphicx}
\usepackage{color}

\title{Potential Outcomes}
\author{\small Nick Eubank}
\date{\vspace*{.3in} \date}


% This is the beginning of a real document!
\begin{document}


\begin{frame}[c]
\maketitle
\end{frame}

\begin{frame}
  \frametitle{Potential Outcomes}
\pause For a unit of analysis $i$, we WANT to compare:
\begin{itemize}
  \item outcome $y_i$ under treatment $t=1$ (denoted $y_{i, t=1}$) to
  \item outcome $y_i$ under no treatment $t=0$ (denoted $y_{i, t=0}$).
\end{itemize}
\pause We call these the \alert{potential outcomes} for $i$ under different treatments.\\
\pause In an ideal world, we'd call \alert{$\delta = y_{i, t=1} - y_{i, t=0}$} our causal estimate.
\begin{itemize}
  \pause \item \emph{Counter-factual model of causality}
\end{itemize}
\end{frame}


\begin{frame}[c]{Potential Outcomes Framework}
... but we can't see both $y_{i, t=1}$ and $y_{i, t=0}$. Each person can only experience one outcome.

So we'll do two things. First, let's move to populations. Ideally we want:
\begin{eqnarray}
  E(\delta) &=& E(Y_{T=1} - y_{T=0}) \nonumber \\
             &=& E(Y_{T=1}) - E(y_{T=0}) \nonumber
\end{eqnarray}
\pause Called \emph{Average Treatment Effect}, or \emph{ATE}
\end{frame}

\begin{frame}[c]{Potential Outcomes Framework}
But we \emph{still} can't actually see ATE. What we \emph{can} see is:
\begin{eqnarray}
\widehat{ATE} = E(Y_{T=1}|D=1) - E(Y_{T=0}|D=0) \nonumber
\end{eqnarray}
where $D\in\{0,1\}$ tell us whether a given observation \emph{actually} experienced the treatment or not.\\
\pause
\vspace{0.1cm}
Two concepts:
\begin{itemize}
  \item $T\in{0,1}$: \emph{Potential} states of the world.
  \item $D\in{0,1}$: \emph{Actual} populations of people.
\end{itemize}
\end{frame}

\begin{frame}[c]{Potential Outcomes Framework}
What we \emph{want} is for $\widehat{ATE} = ATE$. When is that true?

  \begin{eqnarray}
  \widehat{ATE} &=&E(Y_{T=1}|D=1) - E(Y_{T=0}|D=0) \nonumber \\
                \pause &=&E(Y_{T=1}|D=1) - E(Y_{T=0}|D=0) + \nonumber \\
                &&E(Y_{T=0}|D=1) - E(Y_{T=0}|D=1)  \nonumber \\
               \pause &=& \underbrace{E(Y_{T=1}|D=1) - E(Y_{T=0}|D=1)}_{\text{Avg Treatment on the Treated}} + \nonumber \\ &&\underbrace{E(Y_{T=0}|D=1) - E(Y_{T=0}|D=0)}_{\text{Baseline Difference}} \nonumber
  \end{eqnarray}
\end{frame}

\begin{frame}[c]{Potential Outcomes}
$\underbrace{E(Y_{T=1}|D=1) - E(Y_{T=0}|D=1)}_{\text{Treatment on the Treated}} + \underbrace{E(Y_{T=0}|D=1) - E(Y_{T=0}|D=0)}_{\text{Baseline Difference}}$ \\
\vspace*{1cm}
\emph{Baseline Difference:} Absent treatment, would those who actually got treatment have turned out the same as those who hadn't received treatment.
\end{frame}

\begin{frame}[c]{Potential Outcomes Framework}
$\underbrace{E(Y_{T=1}|D=1) - E(Y_{T=0}|D=1)}_{\text{Avg Treatment on the Treated}} + \underbrace{E(Y_{T=0}|D=1) - E(Y_{T=0}|D=0)}_{\text{Baseline Differences}}$ \\
\vspace*{1cm}
\emph{Treatment on the Treated:} What we measure. This is equal to Average Treatment effect iff
\begin{eqnarray}
E(Y_{T=1}|D=1) - E(Y_{T=0}|D=1) &=& E(Y_{T=1}|D=0) - E(Y_{T=0}|D=0)\nonumber
\end{eqnarray}
in which case 
\begin{eqnarray}
E(Y_{T=1}|D=1) - E(Y_{T=0}|D=1) = E(Y_{T=1}) - E(Y_{T=0})
\end{eqnarray}
In other words, $ATT = ATE$ if the response to treatment of people for whom $D=1$ is the same as that of those for whom $D=0$.
\end{frame}

\begin{frame}[c]{Potential Outcomes Framework}
What we estimate is equivalent to $ATE = E(Y_{T=1}) - E(Y_{T=0})$ if:

\begin{enumerate}
  \item No baseline difference (absent treatment, same outcomes)
  \item Same treatment response (no difference in how treated and untreated would respond if treated)
\end{enumerate}
$\Rightarrow$ Both groups to have \alert{same potential outcomes}
\end{frame}

\begin{frame}[c]{Potential Outcomes Framework}
Suppose we measured the effect of an exercise program on health by just comparing health of people in an exercise class with health of people not in an exercise class. \\
How might these two things have been violated:
\begin{enumerate}
  \item No baseline difference
  \item Same treatment response
\end{enumerate}
\end{frame}


\begin{frame}[c]{Potential Outcomes Framework}
Suppose we measured the effect of advertising on sales by correlating sales with advertising expenditures. \\

How might these two things have been violated:
  \begin{enumerate}
    \item No baseline difference
    \item Same treatment response
  \end{enumerate}
\end{frame}


\end{document}
