% So we make this "beamer" rather than document!

\documentclass[11pt]{beamer}
% For handout add ,handout after 11pt

\usetheme[sectionpage=none,numbering=none]{metropolis}           % Use metropolis theme
	% To do printouts, add ", handout"  after aspectratio.
\usepackage{booktabs}
\usepackage{graphicx}
\usepackage{color}
\usepackage{ulem}
\title{What You Can Do}
\author{\small Nick Eubank}
\date{\vspace*{.3in} \date}


% This is the beginning of a real document!
\begin{document}


\begin{frame}[c]
\maketitle
\end{frame}


\begin{frame}[c]
    \centering
    \textbf{Modeling} \\
    Developing Model to Faithfully Represent Data \\
    \vspace*{0.2cm}
    \pause $\Downarrow$ \\
    \vspace*{0.2cm}
     \textbf{Inference} \\
    Interpreting Model Parameters \\
    for Application \\
    \vspace{1cm}
    \begin{enumerate}
        \pause \item Are my estimates causal?
        \pause \item Will they generalize?
        \pause \item Do they reflect biases in an unethical way?
    \end{enumerate}
\end{frame}



\begin{frame}[c]{Day 1}
    \pause By the end of this course, you will:
        \begin{itemize}
            \item Understand why causal inference is hard, \\
            \uncover<3->{{\color{gray} \small Potential Outcomes framework, learned to apply it in a wide range of contexts.}}
            \item Be able to critically evaluate causal evidence collected by others, \\
            \uncover<4->{{\color{gray} \small Evaluated and learned to respond to claims about insurance, advertising, health, and the effects of products}}
            \item Articulate causal questions, \\
            \uncover<5->{{\color{gray} \small  Convert vague stakeholder prompts into actionable questions}}
            \item And develop research designs to answer those questions. \\
            \uncover<6->{{\color{gray} \small Developing, along the way, great portfolio pieces!}}
        \end{itemize}
\end{frame}
    

% \begin{frame}[c]{Causal Inference}
%     \only<1>{``Correlation does not  imply causation''}
%     \only<2->{\sout{``Correlation does not imply causation''}} \\
%     \vspace*{0.2cm}
%     \pause Correlation does not \emph{necessarily} imply causation, \alert{but...}
%     \begin{itemize}
%       \pause \item when certain assumptions are met, correlation does imply causation. 
%     \end{itemize} 
%     \pause And now \alert{you} know those assumptions and how to evaluate them!
% \end{frame}
  

% \begin{frame}[c]{Causal Inference}
%     \pause Causal inference is \alert{hard}.
% \end{frame}
  
% \begin{frame}[c]{Causal Inference}
% Causal inference is hard because of the \alert{Fundamental Problem of Causal Inference:} \\

% \pause To \alert{know} if X causes Y, we would have to see \alert{both} a world with X, and a world without X, and that's impossible. 
% \end{frame}


% \begin{frame}[c]{Causal Inference}
%     Because we can never see \alert{both} a world with X, and a world without X, \pause we need to find settings that \alert{approximate} one of these states of the world. \\
%     \begin{itemize}
%       \pause \item Counter-factuals: settings with same \alert{potential outcomes}, but different realizations of treatment.
%     \end{itemize}
% \end{frame}

\begin{frame}[c]{Ways of Finding Good Counter-Factuals}
    \begin{enumerate}
      \pause \item Randomized Control Trials \\
      {\color{gray} Law of large numbers $\rightarrow$ same potential outcomes for C \& T}
      \pause \item Regression \\
      {\color{gray} Statistically adjust for baseline differences $\rightarrow$ same potential outcomes after adjustments}
      \pause \item Matching \\
      {\color{gray} Statistically adjust for baseline differences $\rightarrow$ same potential outcomes after adjustments}
      \pause \item Differences-in-Differences \\
      {\color{gray} Adjust for pre-existing baseline differences $\rightarrow$ same potential outcomes in trends}
    \end{enumerate}
\end{frame}


\begin{frame}[c]{Cultivate a Skeptical Mindset}
\pause There are \alert{no statistical tests} that can tell you if a causal estimate is valid. 
\begin{itemize}
    \pause \item Depending on \alert{how} data was collected, the exact same dataset may or may not provide valid causal estimates.
\end{itemize}
\pause There are \alert{no statistical tests} that can tell you if a model is behaving in an ethical manner.
\begin{itemize}
    \pause \item The same model may be ethical when used for one purpose, but not in another.
\end{itemize}
\end{frame}


\begin{frame}[t]{You haven't just \emph{learned} about these...}
    \pause You've done them!
    \begin{itemize}
        \pause \item  Used randomization and A/B test data to measure effect of website design changes and racial discrimination in hiring.
        \pause \item  Used regression to estimate gender wage discrimination in the US
        \pause \item  Used fixed effects to understand tax policy's impact on car accidents
        \pause \item  Used matching to estimate labor market returns to education
        \pause \item  Used diff-in-diffs to understand impact of drug legalization on crime
    \end{itemize}
\end{frame}

\begin{frame}[t]{You haven't just \emph{learned} about these...}
You've done them! \\
\pause 
In your projects you've studied the effects of:
\begin{itemize}
    \item militant activity on women's education,
    \item changes in labor laws on wages,
    \item election results on hate crimes,
    \item changes in policing practices on police violence and arrests,
    \item sex education on teenage pregnancy,
    \item school policy on covid rates, 
    \item tournament tennis game duration on future success,
    \item renewable targets on CO2 emissions,
    \item pandemics on stress and dreaming, and
    \item AirBnB designations on earnings.
\end{itemize}
\pause And you did it all while getting used to dealing with \alert{people} again.
\pause And while doing \alert{Kyle's class}.
\end{frame}
    

\begin{frame}[c]
\pause 
\centering
      \includegraphics[width=0.8\textwidth]{maybe_the_real_treasure.png}
\end{frame}

\begin{frame}[c]
    \centering 
    \alert{I hope you feel really proud of everything you've accomplished.}
\end{frame}
    


\end{document}
