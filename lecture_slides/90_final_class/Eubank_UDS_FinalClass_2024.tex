% So we make this "beamer" rather than document!

\documentclass[11pt]{beamer}
% For handout add ,handout after 11pt

\usetheme[sectionpage=none,numbering=none]{metropolis}           % Use metropolis theme
	% To do printouts, add ", handout"  after aspectratio.
\usepackage{booktabs}
\usepackage{graphicx}
\usepackage{color}
\usepackage{ulem}
\title{What You Can Do}
\author{\small Nick Eubank}
\date{\vspace*{.3in} \date}


% This is the beginning of a real document!
\begin{document}


\begin{frame}[c]
\maketitle
\end{frame}

\begin{frame}[c]
    \centering
    \textbf{Modeling and Representation of Data} \\
    Develop Models to Faithfully Represent Data \\
    \vspace*{0.2cm}
    \pause \textbf{Practical Data Science} \\
    Wrestle Data into Workable Forms \\
    \vspace*{0.2cm}
    \pause \textbf{Natural Language Processing} \\
    Convert Text into Machine Interpretable Data \\
    \vspace*{0.2cm}
    \pause \textbf{Data Engineering} \\
    Learned Docker  \\
    \vspace*{0.2cm}
    \pause \textbf{Machine Learning} \\
    Learned the Fit and Evaluate Machine Learning Models \\
    \vspace*{0.2cm}
    \pause $\Rightarrow$ Your toolboxes are now \alert{phenomenally powerful.}
\end{frame}


\begin{frame}[c]
But we don't do data science to accumulate tools (as much fun as they are). \\
\pause We do data science to \alert{solve problems.} \\
\end{frame}


\begin{frame}[c]
\pause This semester, we've learned to:
\begin{itemize}
    \pause \item Identify \alert{core problems},
    \begin{itemize}
        \pause \item (managing and working with stakeholders)
    \end{itemize}
    \pause \item Articulate and refine a question to answer,
    \pause \item Recognize the purpose of different \alert{types of questions},
    \pause \item And then use the \alert{appropriate} amazing tool to generate an answer.
\end{itemize}
\end{frame}

\begin{frame}[c]
I recognize that at the moment, parts of this process may feel... unnecessary or superfluous. \\
\vspace{0.2cm}

\pause $\Rightarrow$ As you move into your capstones, and eventually into the real world, you'll find that proficiency in these skills is what determines the success (or failure) of your projects. \\ 
\vspace{0.2cm}
\pause $\Rightarrow$ These are also the critical reasoning skills that won't be supplanted by chatGPT any time soon.
\end{frame}

\begin{frame}[c]{Causal Inference}
\end{frame}

\begin{frame}[c]{Causal Inference}
    \only<1>{``Correlation does not imply causation''}
    \only<2->{\sout{``Correlation does not imply causation''}} \\
    \vspace*{0.2cm}
    \pause Correlation does not \emph{necessarily} imply causation, \alert{but...}
    \begin{itemize}
      \pause \item when certain assumptions are met, correlation does imply causation. 
    \end{itemize} 
    \pause And now \alert{you} know those assumptions and how to evaluate them!
\end{frame}
  
% \begin{frame}[c]{Causal Inference}
% Causal inference is hard because of the \alert{Fundamental Problem of Causal Inference:} \\

% \pause To \alert{know} if X causes Y, we would have to see \alert{both} a world with X, and a world without X, and that's impossible. 
% \end{frame}


% \begin{frame}[c]{Causal Inference}
%     Because we can never see \alert{both} a world with X, and a world without X, \pause we need to find settings that \alert{approximate} one of these states of the world. \\
%     \begin{itemize}
%       \pause \item Counter-factuals: settings with same \alert{potential outcomes}, but different realizations of treatment.
%     \end{itemize}
% \end{frame}


\begin{frame}[c]{Cultivate a Skeptical Mindset}
    \pause There are \alert{no statistical tests} that can tell you if a causal estimate is valid. 
    \begin{itemize}
        \pause \item Depending on \alert{how} data was collected, the exact same dataset may or may not provide valid causal estimates.
    \end{itemize}
    \end{frame}
    
\begin{frame}[c]{Ways of Estimating Causal Effects}
    \begin{enumerate}
      \pause \item Randomized Control Trials \\
      {\color{gray} Law of large numbers $\rightarrow$ same potential outcomes for C \& T}
      \pause \item Regression \\
      {\color{gray} Statistically adjust for baseline differences $\rightarrow$ same potential outcomes after adjustments}
      \pause \item Matching \\
      {\color{gray} Statistically adjust for baseline differences $\rightarrow$ same potential outcomes after adjustments}
      \pause \item Differences-in-Differences \\
      {\color{gray} Adjust for pre-existing baseline differences $\rightarrow$ same potential outcomes in trends}
    \end{enumerate}
\end{frame}


\begin{frame}[t]{You haven't just \emph{learned} about these...}
    \pause You've done them!
    \begin{itemize}
        \pause \item  Used randomization and A/B test data to measure effect of website design changes and racial discrimination in hiring.
        \pause \item  Used regression to estimate gender wage discrimination in the US
        \pause \item  Used fixed effects to understand tax policy's impact on car accidents
        \pause \item  Used matching to estimate labor market returns to education
        \pause \item  Used diff-in-diffs to understand impact of drug legalization on crime
    \end{itemize}
\end{frame}


\begin{frame}[t]{You haven't just \emph{learned} about these...}
You've done them! \\
\pause 
In your projects you're studying the effects of:
\begin{itemize}
    \item stay at home orders on the spread of covid,
    \item fiscal incentives on policing,
    \item open carry laws on violent crime,
    \item investment in housing infrastructure on homelessness,
    \item childhood trauma on depression,
    \item remote learning on student mental health,
    \item the death penalty on violent crime, 
    \item the pandemic on student learning,
    \item patent protections on drug price growth, and
    \item same sex marriage laws on hate crimes.
\end{itemize}
\pause And you did it all while learning to be more effective colleagues.
\pause And while doing \alert{Kyle's class}.
\end{frame}
    

\begin{frame}[c]
\pause 
\centering
      \includegraphics[width=0.8\textwidth]{maybe_the_real_treasure.png}
\end{frame}

\begin{frame}[c]
    \centering 
    \alert{I hope you feel really proud of everything you've accomplished.}
\end{frame}


\end{document}
