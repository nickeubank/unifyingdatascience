% So we make this "beamer" rather than document!

\documentclass[11pt]{beamer}
% For handout add, handout after 11pt

\usetheme[sectionpage=none,numbering=none]{metropolis}           % Use metropolis theme
	% To do printouts, add ", handout"  after aspectratio.
\usepackage{booktabs}
\usepackage{graphicx}
\usepackage{color}

\title{Team Debriefs}
\author{\small Nick Eubank}
\date{\vspace*{.3in} \date}


% This is the beginning of a real document!
\begin{document}


\begin{frame}[c]
\maketitle
\end{frame}

\begin{frame}[c]{Psychological Safety}
\begin{figure}
    \includegraphics[width=\textwidth]{team_types.png}
\end{figure}
\end{frame}


\begin{frame}[c]{Debrief/After Action Report}
    \begin{itemize}
        \item Constructive effort to learn and iterate
        \item Critical that lessons can be quickly put into action
        \begin{itemize}
            \item Thursday, we'll meet in our new teams and write Team Charters!
        \end{itemize}
    \end{itemize}
\end{frame}

\begin{frame}[c]{Today}
\begin{enumerate}
    \item Identify things you think went \emph{well} with your team. 
    \item Discuss \alert{why} you think they went well?
    \item Identify things you think could have gone better.
    \item Discuss \alert{how} you think you could avoid similar problems in the future?
\end{enumerate}
\end{frame}

\begin{frame}[c]{Today}
As you discuss:
\begin{itemize}
    \item Start by going around the group and letting each person speak, uninterrupted, for around a minute. Then after everyone has had an opportunity to speak, start processing the issues brought up.
    \item Focus on solutions (``How can we work toward making sure this goes more smoothly next time?'', ``What can we do together to make a game plan for next time?'')
    \item Be honest and if you have concerns, speak up. At the same time, however, remember to speak to \emph{your} experiences and remember to be \alert{generous and cautious} in how you interpret the actions of others.
\end{itemize}
\end{frame}
\end{document}
